%%%%%%%%%%%%%%%%%
% This is an sample CV template created using altacv.cls
% (v1.3, 10 May 2020) written by LianTze Lim (liantze@gmail.com). Now compiles with pdfLaTeX, XeLaTeX and LuaLaTeX.
%
%% It may be distributed and/or modified under the
%% conditions of the LaTeX Project Public License, either version 1.3
%% of this license or (at your option) any later version.
%% The latest version of this license is in
%%    http://www.latex-project.org/lppl.txt
%% and version 1.3 or later is part of all distributions of LaTeX
%% version 2003/12/01 or later.
%%%%%%%%%%%%%%%%

%% If you are using \orcid or academicons
%% icons, make sure you have the academicons
%% option here, and compile with XeLaTeX
%% or LuaLaTeX.
% \documentclass[10pt,a4paper,academicons]{altacv}

%% Use the "normalphoto" option if you want a normal photo instead of cropped to a circle
% \documentclass[10pt,a4paper,normalphoto]{altacv}

\documentclass[10pt,letter,ragged2e,withhyper]{altacv}
%% AltaCV uses the fontawesome5 and academicons fonts
%% and packages.
%% See http://texdoc.net/pkg/fontawesome5 and http://texdoc.net/pkg/academicons for full list of symbols. You MUST compile with XeLaTeX or LuaLaTeX if you want to use academicons.

% Change the page layout if you need to
\geometry{left=1.2cm,right=1.2cm,top=1.2cm,bottom=1.2cm,columnsep=1cm}

% The paracol package lets you typeset columns of text in parallel
\usepackage{paracol}

% Change the font if you want to, depending on whether
% you're using pdflatex or xelatex/lualatex
\ifxetexorluatex
  % If using xelatex or lualatex:
  \setmainfont{Roboto Slab}
  \setsansfont{Lato}
  \renewcommand{\familydefault}{\sfdefault}
\else
  % If using pdflatex:
  \usepackage[rm]{roboto}
  \usepackage[defaultsans]{lato}
  % \usepackage{sourcesanspro}
  \renewcommand{\familydefault}{\sfdefault}
\fi

% Change the colours if you want to
\definecolor{SlateGrey}{HTML}{2E2E2E}
\definecolor{LightGrey}{HTML}{666666}
\definecolor{DarkPastelRed}{HTML}{450808}
\definecolor{PastelRed}{HTML}{8F0D0D}
\definecolor{GoldenEarth}{HTML}{E7D192}
\colorlet{name}{black}
\colorlet{tagline}{PastelRed}
\colorlet{heading}{DarkPastelRed}
\colorlet{headingrule}{GoldenEarth}
\colorlet{subheading}{PastelRed}
\colorlet{accent}{PastelRed}
\colorlet{emphasis}{SlateGrey}
\colorlet{body}{LightGrey}

% Change some fonts, if necessary
\renewcommand{\namefont}{\Huge\rmfamily\bfseries}
\renewcommand{\personalinfofont}{\footnotesize}
\renewcommand{\cvsectionfont}{\LARGE\rmfamily\bfseries}
\renewcommand{\cvsubsectionfont}{\large\bfseries}


% Change the bullets for itemize and rating marker
% for \cvskill if you want to
\renewcommand{\itemmarker}{{\small\textbullet}}
\renewcommand{\ratingmarker}{\faCircle}

%% sample.bib contains your publications
\addbibresource{sample.bib}

\begin{document}
\name{Ryan Soedjak}
\tagline{}
%% You can add multiple photos on the left or right
%\photoR{2.8cm}{Globe_High}
% \photoL{2.5cm}{Yacht_High,Suitcase_High}

\personalinfo{%
  % Not all of these are required!
  \email{soedjak.ryan@gmail.com}
  \phone{+1 (417) 319-8687}
  %\mailaddress{Åddrésş, Street, 00000 Cóuntry}
  \location{St. Louis, USA}
  %\homepage{www.homepage.com}
  %\twitter{@twitterhandle}
  \linkedin{rsoedjak}
  \github{el-ectric}
  %% You MUST add the academicons option to \documentclass, then compile with LuaLaTeX or XeLaTeX, if you want to use \orcid or other academicons commands.
  % \orcid{0000-0000-0000-0000}
  %% You can add your own arbtrary detail with
  %% \printinfo{symbol}{detail}[optional hyperlink prefix]
  % \printinfo{\faPaw}{Hey ho!}[https://example.com/]
  %% Or you can declare your own field with
  %% \NewInfoFiled{fieldname}{symbol}[optional hyperlink prefix] and use it:
  % \NewInfoField{gitlab}{\faGitlab}[https://gitlab.com/]
  % \gitlab{your_id}
}

\makecvheader
%% Depending on your tastes, you may want to make fonts of itemize environments slightly smaller
% \AtBeginEnvironment{itemize}{\small}

%% Set the left/right column width ratio to 6:4.
\columnratio{0.6}

% Start a 2-column paracol. Both the left and right columns will automatically
% break across pages if things get too long.
\begin{paracol}{2}
\cvsection{Education}

\unievent{Candidate for B.S. in Computer Science, GPA: 4.00}{Georgia Institute of Technology}{August 2017--Present}{Atlanta, GA}{Expected Graduation: May 2021}

\studyabroadevent{Hong Kong University of Science and Technology}{Fall 2019}{Hong Kong}
\studyabroadevent{Southern University of Science and Technology }{Summer 2018}{Shenzhen, China}

\cvsection{Experience}

\cvevent{AWS Software Development Intern}{Amazon}{Summer 2020}{Virtual Remote}
\begin{itemize}
\item Created Java database tool that interfaces with internal AWS to help engineers debug problems in Cloudfront edge locations
\end{itemize}

\divider

\cvevent{Course Grader}{Georgia Institute of Technology}{Summer 2019}{Atlanta, GA}
\begin{itemize}
\item Graded assignments and exams for Georgia Tech’s MATH 3215: Probability \& Statistics
\end{itemize}

% \divider

% \cvevent{Coursework Grader}{Art of Problem Solving}{February 2017--May 2019}{Online}
% \begin{itemize}
% \item Guided students on how to improve solution writing in Number Theory, Geometry, Counting \& Probability
% \end{itemize}

\cvsection{Projects}

\cvevent{CADe}{Georgia Institute of Technology}{Fall 2020--Present}{}
\begin{itemize}
\item Currently developing interconnected Android, iOS, and Windows tools to increase user input bandwidth when using CAD software
\item Performed extensive market research by interviewing over 100 engineers; Developed MVP; Filed a provisional patent
\item Won Georgia Tech CreateX Capstone; Accepted into Georgia Tech StartUp Launch with \$4000 grant; Advanced to Inventure Prize Semi-finals
\end{itemize}

\divider

\cvevent{Georgia Tech ME 2110 Mechatronics Robot}{Georgia Institute of Technology}{Fall 2018}{}
\begin{itemize}
\item Built a robot that competes against others to execute certain mechanical tasks in an arena
\item In charge of prototyping, electrical components, programming, and report writing
\item Placed 5th out of 59 teams
\end{itemize}

\medskip

% use ONLY \newpage if you want to force a page break for
% ONLY the current column
\newpage

% \cvsection{Publications}

% \nocite{*}

% \printbibliography[heading=pubtype,title={\printinfo{\faBook}{Books}},type=book]

% \divider

% \printbibliography[heading=pubtype,title={\printinfo{\faFile*[regular]}{Journal Articles}},type=article]

% \divider

% \printbibliography[heading=pubtype,title={\printinfo{\faUsers}{Conference Proceedings}},type=inproceedings]

%% Switch to the right column. This will now automatically move to the second
%% page if the content is too long.
\switchcolumn

\cvsection{Skills}

\cvsubsection{Programming \& Software}
\cvtag{Python}
\cvtag{Java}
\cvtag{MATLAB}
\cvtag{SQL}\\
\cvtag{HTML}
\cvtag{CSS}
\cvtag{NI LabVIEW}
\cvtag{IDA Pro}
\cvtag{Ollydbg}
\cvtag{LaTeX}
\cvtag{Solidworks}

\medskip
\divider

\cvsubsection{Relevant Coursework}
\cvtag{Machine Learning}
\cvtag{Deep Learning}\\
\cvtag{Game AI}
\cvtag{Advanced Algorithms}\\
\cvtag{Database Systems}\\
\cvtag{Reverse Malware Engineering}

\cvtag{Linear Algebra}
\cvtag{Abstract Algebra}\\
\cvtag{Probability \& Statistics}
\cvtag{Number Theory}
\cvtag{Game Theory}

\medskip
\divider

\cvsubsection{Languages}
\cvtag{English (Fluent)}
\cvtag{Indonesian (Beginner)}
\cvtag{Japanese (Beginner)}

\medskip
\cvsection{Leadership}

\cvevent{Volunteer Math Tutor}{}{2013--2017}{Springfield, MO}
\begin{itemize}
\item Taught Algebra, Number Theory, Combinatorics, Geometry, Trigonometry, and Calculus to local middle and high school students to prepare for math competitions
\end{itemize}


\cvsection{Competition Awards}

\cvachievement{\faTrophy}{Google Games Atlanta}{2018: 4th Place Team}

\divider

\cvachievement{\faTrophy}{Mercer Spring Programming Competition}{2018: 6th Place Team}

\divider

\cvachievement{\faTrophy}{William Lowell Putnam Competition}{2017: Top 30\%}



%% Yeah I didn't spend too much time making all the
%% spacing consistent... sorry. Use \smallskip, \medskip,
%% \bigskip, \vpsace etc to make ajustments.
%\medskip


% \divider

% \cvsection{Referees}

% % \cvref{name}{email}{mailing address}
% \cvref{Prof.\ Alpha Beta}{Institute}{a.beta@university.edu}
% {Address Line 1\\Address line 2}

% \divider

% \cvref{Prof.\ Gamma Delta}{Institute}{g.delta@university.edu}
% {Address Line 1\\Address line 2}


\end{paracol}


\end{document}
